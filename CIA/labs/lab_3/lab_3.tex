%%% Template originaly created by Karol Kozioł (mail@karol-koziol.net) and modified for ShareLaTeX use

\documentclass[a4paper,11pt]{article}

\usepackage[T1]{fontenc}
\usepackage[utf8]{inputenc}
\usepackage{graphicx}
\usepackage{xcolor}
\usepackage{amsmath,amssymb,amsthm}
\usepackage{enumerate}
\usepackage{multicol}
\usepackage{tikz}
\usepackage[force,almostfull]{textcomp}
\usepackage{geometry}

\geometry{total={210mm,297mm},
left=25mm,right=25mm,%
bindingoffset=0mm, top=20mm,bottom=20mm}

\renewcommand*\sfdefault{phv}
\renewcommand\familydefault{\sfdefault}

\newcommand*{\TitleFont}{%
      \usefont{\encodingdefault}{\rmdefault}{b}{n}%
      \fontsize{16}{20}%
      \selectfont}


\linespread{1.3
}
\newcommand{\linia}{\rule{\linewidth}{0.5pt}}

% custom theorems if needed
\newtheoremstyle{mytheor}
    {1ex}{1ex}{\normalfont}{0pt}{\scshape}{.}{1ex}
    {{\thmname{#1 }}{\thmnumber{#2}}{\thmnote{ (#3)}}}

\theoremstyle{mytheor}
\newtheorem{defi}{Definition}

% my own titles
\makeatletter
\renewcommand{\maketitle}{
\begin{center}
\vspace{2ex}
{\huge \textsc{\@title}}
\vspace{1ex}
\\
\linia\\
\@author \hfill \@date
\vspace{4ex}
\end{center}
}
\makeatother
%%%

% custom footers and headers
\usepackage{fancyhdr}
\pagestyle{fancy}
\lhead{}
\chead{}
\rhead{}
\lfoot{Assignment \#3: DNS(1) }
\cfoot{}
\rfoot{Page \thepage}
\renewcommand{\headrulewidth}{0pt}
\renewcommand{\footrulewidth}{0pt}
%

% code listing settings
\usepackage{listings}
\lstset{
    % language=Bash,
    basicstyle=\ttfamily\small,
    aboveskip={1.0\baselineskip},
    belowskip={1.0\baselineskip},
    columns=fixed,
    extendedchars=true,
    breaklines=true,
    tabsize=4,
    prebreak=\raisebox{0ex}[0ex][0ex]{\ensuremath{\hookleftarrow}},
    frame=lines,
    showtabs=false,
    showspaces=false,
    showstringspaces=false,
    keywordstyle=\color[rgb]{0.627,0.126,0.941},
    commentstyle=\color[rgb]{0.133,0.545,0.133},
    stringstyle=\color[rgb]{01,0,0},
    numbers=left,
    numberstyle=\small,
    stepnumber=1,
    numbersep=10pt,
    captionpos=t,
    escapeinside={\%*}{*)}
}

%%%----------%%%----------%%%----------%%%----------%%%

\begin{document}

\title{\TitleFont Assignment \#3: DNS(1) }

\author{Emil Sharifulllin, Innopolis University}

\date{\today}

\maketitle

\section{Introduction}
The nameserver was deployed at Ubuntu:16.04 docker container: 

\begin{lstlisting}
$ cat /etc/*release
DISTRIB_ID=Ubuntu
DISTRIB_RELEASE=16.04
DISTRIB_CODENAME=xenial
DISTRIB_DESCRIPTION="Ubuntu 16.04.1 LTS"
NAME="Ubuntu"
VERSION="16.04.1 LTS (Xenial Xerus)"
ID=ubuntu
ID_LIKE=debian
PRETTY_NAME="Ubuntu 16.04.1 LTS"
VERSION_ID="16.04"
HOME_URL="http://www.ubuntu.com/"
SUPPORT_URL="http://help.ubuntu.com/"
BUG_REPORT_URL="http://bugs.launchpad.net/ubuntu/"
UBUNTU_CODENAME=xenial
\end{lstlisting}


\section{Downloading and Installing a Caching Nameserver}
Nameserver can be downloaded from site http://www.isc.org by following command:
\begin{lstlisting}
$ wget -O bind.tar.gz https://www.isc.org/downloads/file/bind-9-10-4-p2/?version=tar-gz
\end{lstlisting}

To unarchieve tarbal we can use following command:
\begin{lstlisting}
$ tar -xzvf bind.tar.gz
\end{lstlisting}

\subsection{Validating the Download}
Download can ve validated by GPG with ASC file. To download this file and check package we can use following bash command.

\begin{lstlisting}
$ wget -O bind.tar.gz.asc ftp://ftp.isc.org/isc/bind9/9.10.4-P2/bind-9.10.4-P2.tar.gz.asc
$ gpg --verify bind.tar.gz.asc bind.tar.gz
gpg: Signature made Mon Jul 18 22:59:45 2016 UTC using RSA key ID 911A4C02
gpg: Good signature from "Internet Systems Consortium, Inc. (Signing key, 2015-2016) <codesign@isc.org>"
\end{lstlisting}

\textbf{Why is it wise to use a signature to check your download?}\\
Signing is very important because after checking signature you can be sure that certain tarbal is right and no one can spoof this tarbal.

\textbf{Which kind of signature is the best one to use? Why?}\\
PGP signing is the safest way to check package validity because probability of collision of PGP key is lower than probability of collision of SHA key. 

\subsection{Installation Documentation}
Documentation can be found in doc/ directory. README file contains installation guides and is located at root installation directory.

\subsection{Compiling}
To compile bind you need to rum following commands:
\begin{lstlisting}
$ ./configure
$ make
$ make install
\end{lstlisting}

\section{Configuring and Testing}

After installing DNS server we will start to configure it.\\

\textbf{Why are caching-only name servers still useful?}\\
It can store resourse Records during theirs TTL to decrease DNS resolving time and decrease network flood. It can reduce the time needed to internet communication.\\

\subsection{Main Configuration}
To configure BIND as a caching server it is necessary to create file /etc/named.conf . To enable remote control to nameserver we need to generate keys for RNDC.

\begin{lstlisting}
$ rndc-confgen
# Start of rndc.conf
key "rndc-key" {
    algorithm hmac-md5;
    secret "VERY SECRET";
};

options {
    default-key "rndc-key";
    default-server 127.0.0.1;
    default-port 953;
};

\end{lstlisting}

\subsection{Root Servers}
BIND needs list of root servers to work. this list can be downloaded from ftp://ftp.rs.
internic.net/domain. In named.conf we need to define path to root severs cache file.

\begin{lstlisting}
zone "." {
    type hint;
    file "named.root";
};
\end{lstlisting}

\subsection{Resolving localhost}
To enable reverse mapping for loopback address 127.0.0.1 we need to add zone file.
\begin{lstlisting}[]
$TTL 86400
@ IN SOA localhost. admin.localhost. (
1   ; serial
360000  ; refresh every 100 hours
3600    ; retry after 1 hour
3600000 ; expire after 1000 hours
3600    ; negative cache is 1 hour
)

    IN  NS  ns.st10.os3.su.
0   IN  PTR loopback.
1   IN  PTR localhost.
\end{lstlisting}

\textbf{Now that you know all the elements of the main configuration, create
a simple named.conf or unbound.conf for a caching-only name server.
Show the configuration file in your report.}\\

\begin{lstlisting}
//Define a access list to limit recursion later
acl localnet {
    127.0.0.1/32;
};

controls {
    inet 127.0.0.1 port 953
        allow { 127.0.0.1; } keys { "rndc-key"; };
};

key "rndc-key" {
    algorithm hmac-md5;
    secret "xxsGwSWnOTIOvyIbdFjtAQ==";
};


// Working directory and limit recursion
options {
    directory "/etc/bind";
    allow-recursion {
        localnet;
    };
};

// Caching only DNS server
zone "." {
    type hint;
    file "named.root";
};

zone "st10.os3.su." {
    type master;
    file "st10.os3.su.zone";
};

// Provide a reverse mapping for the loopback address 127.0.0.1
zone "0.0.127.in-addr.arpa" {
type master;
file "local.zone";
notify no;
};
\end{lstlisting}

\subsection{Testing}
To test configuration files we can use two checkconf and checkzone programs

\begin{lstlisting}
$ named-checkconf
$ echo $?
0
$ named-checkzone localhost /etc/bind/local.zone
zone localhost/IN: loaded serial 1
OK
\end{lstlisting}

\subsubsection{Testing of cache server}
\begin{lstlisting}
$ dig google.com @127.0.0.1
...
;; Query time: 318 msec
;; SERVER: 127.0.0.1#53(127.0.0.1)
...

$ dig google.com @127.0.0.1
...
;; Query time: 0 msec
;; SERVER: 127.0.0.1#53(127.0.0.1)
...
\end{lstlisting}

\textbf{Why do the programs return a result value?}\\
Returning result value is very useful and can be used in bash scripts.

\section{Running and Improving the Name Server}

\textbf{Show the changes you made to your configuration to allow remote control}\\
To allow rndc tool I added to named.conf following lines:

\begin{lstlisting}[caption=named.conf]
controls {
    inet 127.0.0.1 port 953
        allow { 127.0.0.1; } keys { "rndc-key"; };
};

key "rndc-key" {
    algorithm hmac-md5;
    secret "xxsGwSWnOTIOvyIbdFjtAQ==";
};
\end{lstlisting}

\begin{lstlisting}[caption=rndc.conf]
key "rndc-key" {
    algorithm hmac-md5;
    secret "xxsGwSWnOTIOvyIbdFjtAQ==";
};

options {
    default-key "rndc-key";
    default-server 127.0.0.1;
    default-port 953;
};

\end{lstlisting}

\textbf{What other commands/functions does rndc/unbound-control provide?}
rndc allows you to control nameserver without stopping and restarting nameserver daemon.

\textbf{What do you need to put in resolv.conf (and/or other files) to use your own name server?}\\
Adding namesevers to resolv.conf is the bad way because after restarting network manager will restore previous values of resolv.conf. To configure Ubuntu to use my own nameserver I need to add following strings to /etc/resolvconf/resolv.conf.d/base and then run sudo resolvconf -u

\begin{lstlisting}
nameserver 127.0.0.1
\end{lstlisting}

\section{Configuring an Authoritative Nameserver}

\textbf{Show the forward mapping zone file in your log.}\\
My zone file contain following information:

\begin{lstlisting}
$TTL 86400
@ IN SOA ns10.os3.su. admin.st10.os3.su. (
2016082900  ; serial
360000  ; refresh every 100 hours
3600    ; retry after 1 hour
3600000 ; expire after 1000 hours
3600    ; negative cache is 1 hour
)


@       IN  NS      ns10.os3.su.
@       IN  A       188.130.155.43
@       IN  MX  10  mail
st11    IN  NS      ns11.os3.su.
ns      IN  A       188.130.155.43
www     IN  A       188.130.155.43
mail    IN  A       188.130.155.43
web     IN  CNAME   www
mob     IN  CNAME   www
ns1     IN  CNAME   ns
ns2     IN  CNAME   ns
\end{lstlisting}

\textbf{If Azat had not yet implemented the delegation, what information would you need
to give him so that he can implement it?}\\
To setup delegation for my zone st10.os3.su. Azat must know three things:
\begin{enumerate}
    \item \textbf{My zone:} st10.os3.su
    \item \textbf{My ip address:} 188.130.155.43
    \item \textbf{My nameserver:} ns10.os3.su
\end{enumerate}

\textbf{What important requirement is not yet met for your subdomain?}\\
\begin{enumerate}
    \item We need to setup secondary nameserver to provide fault tolerance.
    \item We need to add SOA record for subdomain and A records for subdomain nameservers.
\end{enumerate}

\end{document}

\grid
