\documentclass[a4paper,11pt]{article}
\usepackage[utf8]{inputenc}
\usepackage{graphicx}
\usepackage{enumerate}
\usepackage{geometry}
\usepackage{fancyhdr}
\usepackage{minted}
\usepackage{xcolor}
\usepackage[colorlinks = true,
            linkcolor = blue,
            urlcolor  = blue,
            citecolor = blue]{hyperref}

\geometry{total={210mm,297mm},
left=25mm,right=25mm,%
bindingoffset=0mm, top=20mm,bottom=20mm}

\graphicspath{ {./images/} }
\renewcommand{\thesubsubsection}{\thesubsection.\alph{subsubsection}}
\renewcommand*\sfdefault{phv}
\renewcommand\familydefault{\sfdefault}

% \renewcommand{\thesubsubsection}{\thesubsection.\alph{subsubsection}}

% \newmintedfile{html}{
%     linenos,
%     breaklines,
%     python3,
%     numbersep=8pt,
%     frame=single,
%     framesep=3mm} 

\newcommand*{\TitleFont}{%
      \usefont{\encodingdefault}{\rmdefault}{b}{n}%
      \fontsize{16}{20}%
      \selectfont}

\linespread{1.3}

% my own titles
\makeatletter
\renewcommand{\maketitle}{
\begin{center}
\vspace{2ex}
{\huge \textsc{\@title}}
\vspace{1ex}
\\
\rule{\linewidth}{0.5pt}\\
\@author \hfill \@date
\vspace{4ex}
\end{center}
}
\makeatother

\definecolor{bg}{rgb}{0.95,0.95,0.95}


% custom footers and headers
\pagestyle{fancy}
\lhead{}
\chead{}
\rhead{}
\lfoot{Assignment 4 : Version control systems/Latex }
\cfoot{}
\rfoot{Page \thepage}
\renewcommand{\headrulewidth}{0pt}
\renewcommand{\footrulewidth}{0pt}
%%----------%%%----------%%%----------%%%----------%%%

\begin{document}


\newmintedfile{tex}{
    linenos,
    breaklines,
    numbersep=8pt,
    frame=single,
    % bgcolor=bg,
    framesep=3mm} 

\newmintedfile{make}{
    linenos,
    breaklines,
    numbersep=8pt,
    frame=single,
    % bgcolor=bg,
    framesep=3mm} 

\newminted{bash}{fontsize=\scriptsize, 
    linenos,
    python3,
    numbersep=8pt,
    frame=single,
    bgcolor=bg,
    framesep=3mm}


% \newminted{all}{linenos, frame=single}

% \usemintedstyle{monokai}
\usemintedstyle{manni}
% \usemintedstyle{xcode}
% \usemintedstyle{vs}
% \usemintedstyle{autumn}
% \usemintedstyle{colorful}
% \usemintedstyle{trac}


\title{ \TitleFont Assignment 4 : Version control systems/Latex }

\author{Emil Sharifulllin, Innopolis University}

\date{\today}

\maketitle

\tableofcontents

\section{LaTeX report}
As my part of work I created file st10.tex and in this file I talked about REST API.

\texfile{./es4/st10.tex}

And I also added this lines to bib file

\begin{bashcode*}{label=team1.bib}
@article{wiki-rest,
title = {REST-API},
journal = {Wikipedia REST API, "\url{https://en.wikipedia.org/wiki/Representational_state_transfer}",},
year = 2016,
}
\end{bashcode*}

\section{Makefile}

In our group we created following Makefile
\makefile{./es4/Makefile}

To make this file you need to type
\begin{bashcode}
$ make
\end{bashcode}

\section{Git repo}
We created git repository with this sequence of actions:

\begin{enumerate}
    \item Create Git repository (created on st1.os3.su)
    \item Initialize git repository on local machines (git init)
    \item Create .gitignore file for saving some space from temporary unneeded files:
    \begin{bashcode}
    build/
    *.pdf
    *.aux
    *.log
    *.out
    *.bbl
    *.blg
    *.lof
    *.synctex.gz
    *.toc
    es4/
    .DS_Store
    *.lot
    *.fdb_latexmk
    *.fls
    *.swp
    _minted*/
    \end{bashcode}
    \item For future convenience create keypair in order to have an ability to connect to git server via ssh (ssh-keygen command).
    \item Send public key to the server 
    \item Create ~/.ssh/config file with contents:
    \begin{bashcode}
    Host bogdan
     HostName 188.130.155.34
     Port 10022
     User git
     IdentityFile ~/.ssh/id_rsa
    \end{bashcode}
    
\end{enumerate}

And you can check this repo with typing:

\begin{bashcode}
$ git clone ssh://git@188.130.155.34:10022/home/git/es4
\end{bashcode}
Password: hiazathereisyourpassword\\

To work with repo I used this commands:

\begin{bashcode}
$ git clone git@bogdan:/home/git/es4
$ git status
$ git add postman.png st10.tex team1.*
$ git commit -m "Emil done his work"
$ git pull
$ git push
\end{bashcode}

\end{document}