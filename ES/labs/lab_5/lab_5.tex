\documentclass[a4paper,11pt]{article}
\usepackage[utf8]{inputenc}
\usepackage{graphicx}
\usepackage{enumerate}
\usepackage{geometry}
\usepackage{fancyhdr}
\usepackage{minted}
\usepackage{xcolor}
\usepackage{listings}
\usepackage[colorlinks = true,
            linkcolor = blue,
            urlcolor  = blue,
            citecolor = blue]{hyperref}

\geometry{total={210mm,297mm},
left=25mm,right=25mm,%
bindingoffset=0mm, top=20mm,bottom=20mm}

\graphicspath{ {./images/} }
\renewcommand{\thesubsubsection}{\thesubsection.\alph{subsubsection}}
\renewcommand*\sfdefault{phv}
\renewcommand\familydefault{\sfdefault}

% \renewcommand{\thesubsubsection}{\thesubsection.\alph{subsubsection}}

% \newmintedfile{html}{
%     linenos,
%     breaklines,
%     python3,
%     numbersep=8pt,
%     frame=single,
%     framesep=3mm} 

\newcommand*{\TitleFont}{%
      \usefont{\encodingdefault}{\rmdefault}{b}{n}%
      \fontsize{16}{20}%
      \selectfont}

\linespread{1.3}

% my own titles
\makeatletter
\renewcommand{\maketitle}{
\begin{center}
\vspace{2ex}
{\huge \textsc{\@title}}
\vspace{1ex}
\\
\rule{\linewidth}{0.5pt}\\
\@author \hfill \@date
\vspace{4ex}
\end{center}
}
\makeatother

\definecolor{bg}{rgb}{0.95,0.95,0.95}


% custom footers and headers
\pagestyle{fancy}
\lhead{}
\chead{}
\rhead{}
\lfoot{Assignment 3 : Version control systems/Latex }
\cfoot{}
\rfoot{Page \thepage}
\renewcommand{\headrulewidth}{0pt}
\renewcommand{\footrulewidth}{0pt}
%%----------%%%----------%%%----------%%%----------%%%

\begin{document}

\newminted{python}{fontsize=\scriptsize, 
    linenos,
    python3,
    numbersep=8pt,
    frame=single,
    bgcolor=bg,
    framesep=3mm} 

\newminted{bash}{fontsize=\scriptsize, 
    linenos,
    python3,
    numbersep=8pt,
    frame=single,
    bgcolor=bg,
    framesep=3mm}


% \newminted{all}{linenos, frame=single}

% \usemintedstyle{monokai}
\usemintedstyle{manni}
% \usemintedstyle{xcode}
% \usemintedstyle{vs}
% \usemintedstyle{autumn}
% \usemintedstyle{colorful}
% \usemintedstyle{trac}


\title{ \TitleFont Assignment 3 : Version control systems/Latex }

\author{Emil Sharifulllin, Innopolis University}

\date{\today}

\maketitle

To create this assignment I chose chose python3 and bottle library.
Result you can see at \href{http://es5.st10.os3.su}{http://es5.st10.os3.su}

\section{API overview}
This API created to manipulate with users and their posts. 

\begin{itemize}
  \item \textbf{GET /users} - list all users
  \item \textbf{GET /users/[login]} - show information about user with certain login
  \item \textbf{GET /users/[login]/posts} - show all posts of user
  \item \textbf{POST /users/[login]/posts} - create post of user
\end{itemize}

\section{Install guides}
To install this app type following commands in terminal

\begin{bashcode}
$ mkdir venv
$ python3 -m venv venv
$ source venv/bin/activate
$ python3 -m pip install -r requirements.txt
$ python3 main.py
\end{bashcode}


\section{Source code}
\begin{pythoncode*}{label=main.py}
#!/usr/bin/env python3

from bottle import run, get, post, abort, request
import sys

USERS = {
    'litleleprikon': {
        'name': 'litleleprikon',
        'surname': 'litleleprikon',
        'posts': [
            {
                'title': 'First post',
                'article': 'My first post on this blog'
            },
            {
                'title': 'First post',
                'article': 'My first post on this blog'
            }
        ]
    },
    'e.sharifullin': {
        'name': 'Emil',
        'surname': 'Sharifullin',
        'posts': [
            {
                'title': 'First post',
                'article': 'My first post on this blog'
            },
            {
                'title': 'First post',
                'article': 'My first post on this blog'
            }
        ]
    }
}

def search_user(func):
    def wrapper(name):
        if name in USERS:
            return func(name)
        else:
            abort(404, "Sorry, Page not found.")
    return wrapper


@get('/')
def hello():
    return {
        'api usage': {
            'GET /': 'this page',
            'GET /users': 'list of users',
            'GET /users/[name]': 'info about user',
            'GET /users/[name]/posts': 'users posts',
            'POST /users/[name]/posts': 'create post for user'
        }
    }

@get('/users')
def users():
    return {'users': list(USERS.keys())}

@get('/users/<name>')
@search_user
def user(name):
    return USERS[name]


@get('/users/<name>/posts')
@search_user
def user_posts(name):
    return {
            'user': name,
            'posts': USERS[name]['posts']
        }


@post('/users/<name>/posts')
@search_user
def user_posts(name):
    if request.json is None:
        abort(400, "Data must be in JSON")
    USERS[name]['posts'].append({
            'title': request.json.get('title'),
            'article': request.json.get('article')
        })
    return {'status': 'Ok'}


def main():
    host = 'localhost'
    port = 8080
    if len(sys.argv) > 3:
        print('Bad number of arguments\nZero or one required')
    if len(sys.argv) > 1:
        host = sys.argv[1]
    if len(sys.argv) == 3:
        port = sys.argv[2]
    run(host='localhost', port=port, debug=True)

if __name__ == '__main__':
    main()

\end{pythoncode*}

\end{document}